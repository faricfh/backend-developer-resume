\documentclass[11pt,a4paper]{article}

% ================== PACKAGES ==================
\usepackage[a4paper,margin=2cm]{geometry}
\usepackage{titlesec}
\usepackage{enumitem}
\usepackage{hyperref}
\usepackage{setspace}
\usepackage{parskip}
\usepackage[T1]{fontenc}
\usepackage{lmodern}

% ================== SETTINGS ==================
\setstretch{1.15}
\setlist[itemize]{leftmargin=*, noitemsep, topsep=4pt}

\titleformat{\section}
  {\Large\bfseries}
  {}
  {0em}
  {}

\titleformat{\subsection}
  {\large\bfseries}
  {}
  {0em}
  {}

% ================== DOCUMENT ==================
\begin{document}

\begin{center}
{\LARGE \textbf{Project Documentation}}\\
\vspace{4pt}
Freelance Development Projects
\end{center}

\vspace{12pt}

% ================== PROJECT 1 ==================
\section*{Insurance Management System}

Sistem manajemen asuransi berbasis web yang dikembangkan untuk kebutuhan administrasi klien.

\subsection*{Deskripsi}
Aplikasi ini digunakan untuk pengelolaan data operasional asuransi, termasuk manajemen pengguna, data polis, dan laporan administrasi.

\subsection*{Fitur Utama}
\begin{itemize}
  \item Dashboard administrasi berbasis peran pengguna
  \item Sistem autentikasi dan kontrol akses
  \item Modul CRUD data operasional
  \item Laporan administrasi berbasis database
\end{itemize}

\subsection*{Teknologi}
Laravel, MySQL, Admin Panel

\vspace{10pt}

% ================== PROJECT 2 ==================
\section*{Escala Seicho-no-ie Management System}

Sistem manajemen jadwal dan pelaporan organisasi berbasis kalender.

\subsection*{Deskripsi}
Aplikasi digunakan untuk mengelola jadwal kegiatan organisasi serta pelaporan aktivitas bulanan.

\subsection*{Fitur Utama}
\begin{itemize}
  \item Sistem penjadwalan berbasis kalender
  \item Manajemen entitas organisasi (nucleo, associacao, orientador)
  \item Ekspor laporan PDF bulanan
  \item Rekap data aktivitas organisasi
\end{itemize}

\subsection*{Teknologi}
Laravel, MySQL, PDF Reporting

\vspace{10pt}

% ================== PROJECT 3 ==================
\section*{Workbench Data Management App}

Aplikasi manajemen data peralatan industri.

\subsection*{Deskripsi}
Digunakan untuk pengelolaan inventaris peralatan industri dengan kontrol akses pengguna.

\subsection*{Fitur Utama}
\begin{itemize}
  \item CRUD data peralatan
  \item Role-based access control
  \item Fitur pencarian data
  \item Ekspor data ke spreadsheet
  \item Sistem autentikasi pengguna
\end{itemize}

\subsection*{Teknologi}
Laravel, MySQL

\vspace{10pt}

% ================== PROJECT 4 ==================
\section*{Valve Equipment Finder}

Aplikasi pencarian peralatan berbasis database.

\subsection*{Deskripsi}
Digunakan untuk membantu pencarian dan pengelolaan data teknis peralatan industri.

\subsection*{Fitur Utama}
\begin{itemize}
  \item Sistem katalog peralatan
  \item Pencarian data cepat
  \item Manajemen data teknis
\end{itemize}

\subsection*{Teknologi}
Laravel, MySQL

\end{document}