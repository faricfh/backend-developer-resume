\documentclass[11pt,a4paper]{article}

% ================== PACKAGES ==================
\usepackage[a4paper,margin=1.8cm]{geometry}
\usepackage{titlesec}
\usepackage{enumitem}
\usepackage{hyperref}
\usepackage{setspace}
\usepackage{parskip}
\usepackage[T1]{fontenc}
\usepackage{lmodern}

% ================== SETTINGS ==================
\setstretch{1.15}
\setlist[itemize]{leftmargin=*, noitemsep, topsep=2pt}

\hypersetup{
  colorlinks=true,
  linkcolor=black,
  urlcolor=black
}

\titleformat{\section}
  {\large\bfseries}
  {}
  {0em}
  {}
  [\titlerule]

% ================== DOCUMENT ==================
\begin{document}

% ================== HEADER ==================
\begin{center}
  {\LARGE \textbf{FARIC FATHIAN HIDAYAH}}\\
  \vspace{4pt}
  Backend Developer\\
  \vspace{6pt}
  Bandung, Jawa Barat \,|\, 
  \href{mailto:faricfathianhidayah@gmail.com}{faricfathianhidayah@gmail.com} \,|\, 
  +62 881-7749-791 \,|\, 
  \href{https://github.com/faricfh}{github.com/faricfh}
\end{center}

\vspace{8pt}

% ================== PROFILE ==================
\section*{Profil Singkat}
Backend Developer dengan pengalaman lebih dari 5 tahun dalam pengembangan dan pemeliharaan aplikasi web.
Terbiasa membangun aplikasi dari awal, menambahkan fitur baru, serta menganalisis dan menyelesaikan permasalahan pada sistem yang sudah berjalan.
Berpengalaman menggunakan Laravel dan Yii2, serta mendukung kebutuhan backend untuk aplikasi mobile sesuai kebutuhan proyek.

% ================== EXPERIENCE ==================
\section*{Pengalaman Kerja}

\textbf{Programmer} \hfill \textit{Juni 2020 -- Sekarang}\\
\textit{CV Putra Purnama Indonesia}
\begin{itemize}
  \item Mengembangkan dan memelihara aplikasi web berbasis Laravel dan Yii2, serta mendukung kebutuhan backend untuk aplikasi mobile.
  \item Menambahkan dan menyesuaikan fitur aplikasi berdasarkan kebutuhan pengguna dan perubahan proses bisnis.
  \item Menganalisis, menelusuri, dan memperbaiki bug untuk menjaga stabilitas dan keandalan sistem.
  \item Menangani permasalahan teknis pada sistem yang sudah berjalan serta melakukan penyesuaian logika aplikasi.
  \item Terlibat dalam pengembangan aplikasi dari tahap awal hingga implementasi produksi.
  \item Menyusun dan menyesuaikan endpoint API sesuai kebutuhan frontend dan aplikasi mobile.
  \item Membuat dan menggunakan Postman collection untuk menguji endpoint API selama proses pengembangan.
  \item Mengelola database MySQL, termasuk relasi tabel dan optimasi query.
  \item Menangani deployment aplikasi ke VPS dan shared hosting.
\end{itemize}

\textbf{PKL / Praktik Kerja Lapangan – Programmer} \hfill \textit{Okt 2019 -- Nov 2019}\\
\textit{Solusi Teknis Bandung}
\begin{itemize}
  \item Mengembangkan aplikasi web sederhana menggunakan PHP dan MySQL.
  \item Membuat fitur CRUD dan antarmuka dasar aplikasi.
  \item Membantu analisis permasalahan sederhana serta perbaikan bug pada aplikasi.
\end{itemize}

% ================== PROJECTS ==================
\section*{Pengalaman Proyek}

\textbf{Sistem Pemerintahan Berbasis Web}
\begin{itemize}
  \item Terlibat dalam pengembangan dan pemeliharaan berbagai sistem internal instansi pemerintah.
  \item Mengerjakan modul Manajemen SDM, Kinerja Pegawai, SOP Digital, serta sistem pendukung administrasi.
  \item Fokus pada pengembangan fitur baru, analisis permasalahan, dan menjaga stabilitas sistem.
\end{itemize}

% ================== SKILLS ==================
\section*{Keahlian Teknis}

\textbf{Backend}
\begin{itemize}
  \item PHP (Laravel, Yii2)
  \item Pengembangan dan pemeliharaan aplikasi backend
\end{itemize}

\textbf{Database \& Tools}
\begin{itemize}
  \item MySQL
  \item Git dan GitHub
  \item API testing dasar menggunakan Postman (request dan collection)
  \item Deployment VPS dan shared hosting
\end{itemize}

\textbf{Pendukung}
\begin{itemize}
  \item Flutter (digunakan pada proyek tertentu)
  \item Next.js (basic)
  \item Kotlin (basic)
\end{itemize}

% ================== EDUCATION ==================
\section*{Pendidikan}

\textbf{SMK Assalaam Bandung} \hfill \textit{Juli 2017 -- Mei 2020}\\
Rekayasa Perangkat Lunak (RPL)
\begin{itemize}
  \item Pengembangan aplikasi berbasis database.
  \item Pengembangan web menggunakan HTML, CSS, PHP, dan Laravel.
  \item Dasar pengembangan aplikasi mobile.
\end{itemize}

\end{document}